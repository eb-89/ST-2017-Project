\documentclass[10pt]{article}

\title{Testing BeautifulSoup - Report}

\author{name1 \\ name2 \\ name3 \\ name4}

\begin{document}

\maketitle

Here is some text that will not be in the report \\

\begin{Description of BeautifulSoup4}
 BeatifulSoup is a library for python which provides functionality to parse HTML and XML files. It provides functionality to search, modify and navigate through a parse tree which the library generats using parsers selected by the user. The library is according to the documentation working similary for both python versions 2.7 and 3.2, we have however limited the scope of our testing to version 2.7. There are previous (and unsupported) versions of BeautifulSoup available, we have limited the scope to the current version 4. The main function of the library generates a nested data structure called a 'soup', by using a parser in combination with the desired markup file. The composition of the object can be used to navigate the tree, as well as explicit functions for searching, traversing and editing the generated soup object. Using a simple loop and a search function a user can easily for example exctact all URLs from a HTML file.
As for the parser, BeatifulSoup supports the HTML parser included in the Python standard library, but the user can also easily install and use external parsers. The documentation notes that BeatifulSoup provides the same interface using different parsers, but if HTML or XML is compared, or if the provided HTML is incorrect, the resulting parse trees will be different depending on the selected parser.
\end{Description of BeautifulSoup4}

lorem ipsum lorem ipsum lorem ipsum lorem ipsum lorem ipsum lorem ipsum lorem ipsum lorem ipsum


\end{document}
